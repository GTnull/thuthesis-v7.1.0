% !TeX root = ../thuthesis-example.tex

\chapter{基于特征关系图的循环神经网络算法}
\section{引言}
从第三章的实验可以知道,现有的异常检测算法能够在公开数据集取得较为不错的表现,但是在真实数据集——清华大学校园网流量数据上效果均无法达到要求。这是因为相比于人工构造的公开数据集,清华大学校园网数据具有规模大、应用类型种类繁多、异常流量占比多、难以学到基线等特点。现有的算法无论是传统机器学习的LR、NB、DT,还是拟合能力更强大的深度学习模型CNN、LSTM、GRU,都有精确率低、误报率高的问题。本章将先从特征分析着手,引入特征之间的关系矩阵,将其加入到神经网络的训练中,从而获得更好的检测效果。

\section{CAMPUS数据集和开源数据集的不同之处}
同样的算法和参数,在不同数据集上表现迥异,并且我们在多个数据集上都尝试了不同的参数。说明数据分布、类别存在差异,。并且变量的维度一致。

通过对比两者的不同,我们发现是因为缺乏特征关系。为此我们将分析特征关系。

下表为UNSW-NB15的攻击类型、每类攻击的占比。三个饼状图。
对攻击类型做个分类,发现不同之处。

寻找攻击类型和特征之间的关系,也就是通过实验发现哪些特征起到了作用。

得出结果:相比于其他两个数据集,CAMPUS数据集具有以下特点:

1.攻击类型更多更复杂。CICIDS数据集仅有7种攻击类型,UNSW-NB有9种。

2.在CAMPUS中特征和攻击类型的关联性更弱,需要进一步挖掘。(如何刻画特征和label之间的关联性)分别画图表示关联性强弱。


使用DNN训练神经网络模型时。Shapley值起源于1953年的博弈论\ref{lundberg2018consistent},是为了解决这样一个问题:一群拥有不同技能的参与者为了集体奖励而相互合作。那么,如何在小组中公平分配奖励?在机器学习领域,参与者就是输入的特征,而集体奖励则是模型输出的结果。Shapley值可用于计算每个特征对于模型输出的贡献。其公式如下: 
\begin{equation}
  \phi_i(v) = \sum_{S\subseteq N \setminus \{i\}} \frac{|S|! (|N| - |S| - 1)!}{|N|!}(v(S \cup \{i\}) - v(S))
\end{equation}

其中,$N$是所有特征构成的集合,$S$是$N$的子集,$v(\cdot)$可以认为是模型,$\phi_i(v)$ 即为第i个特征的“重要性”。

本文利用python中的shap库\footnote{https://github.com/slundberg/shap}分别计算了RNN模型在三个数据集中的Shapley值,并从SHAP特征重要度、SHAP摘要图、SHAP依赖图三方面进行了可视化分析。

SHAP特征重要度的原理是Shapley绝对值越大,其对应的特征越重要。由于公式4-1中计算得到的是单个实例的Shapley值,因此我们需要对每个特征的全部实例取平均值:
\begin{equation}
  I_j = \sum_{i=1}^n |\phi_j^{(i)}|
\end{equation}
接下来我们对每个特征重要度降序排序并且进行绘制,结果如图所示。从中可以看出,排名前列的特征更多是与报文大小相关的特征和报文发送时间相关的特征。通过协议分析得到的特征的重要性相对更弱。

% \begin{figure}
%   \centering
%   \includegraphics[scale=0.1]{mean-sharp-value.bmp}
%   \caption{shap-importance}
%   \label{fig:shap-importance}
% \end{figure}

\begin{figure}
  \centering
  \includegraphics[scale=0.5]{shap-importance.png}
  \caption{shap-importance}
  \label{fig:shap-importance}
\end{figure}

由图中可以看出,在CAMPUS数据集中,特征间重要程度呈现均衡化。

SHAP摘要图将特征影响和特征重要度结合起来。摘要图上的每个点都是一个特征和一个实例的Shapley值,y轴上的位置由特征决定,x轴上的位置由Shapley值决定,颜色代表特征值从小到大,重叠点在y轴方向上抖动,因此我们可以了解每个特征的Shapley值的分布。

\begin{figure}
  \centering
  \includegraphics[scale=0.5]{shap-importance-extended.png}
  \caption{shap-importance-extended}
  \label{fig:shap-importance-extended}
\end{figure}

SHAP依赖图刻画了全部实例的某个特征值和Shapley值之间的关联。

通过本节的分析,我们得出以下结论:1.DNN算法在CAMPUS数据集上失效的原因之一是训练过程中特征重要度相近,没有哪个。2.其他算法的top10特征都是相似的,DNN没有这样。 这也是RNN没有从数据中学习出特征关系的印证。

从两方面分析:分别用同样的算法在不同数据集上对比。
\section{特征分析}
流量特征可以视为反映当前网络流量的一组观察值,从多个角度描述当前的流量场景。好的特征能够真实地反映出流本身的状态信息。例如对于一条单个TCP流,这条流可能是一个正常的端到端的链接,也可能是分布式拒绝服务攻击的一部分,在被攻击者的视角内,它的入流量的带宽急剧增加,且远大于出流量的带宽。图\ref{fig:特征关系}表示某一时刻下部分特征之间的关系,从中我们看出,发送方/接收方数据包长度大小与流的传输速率具有很强的关联性。

\begin{figure}
  \centering
  \includegraphics[scale=1]{特征关系.png}
  \caption{特征关系}
  \label{fig:特征关系}
\end{figure}

为了在神经网络中引入特征关系矩阵,本文利用皮尔逊相关系数来计算流量特征之间的关系。皮尔逊相关的别名是积差相关(或者称之为积矩相关),命名来源是20 世纪英国的描述统计学派先驱皮尔逊。对于两个变量X、Y,它们之间的皮尔逊相关系数为:
% 对于计算变量之间的线性相关性,他想到了一种方法,即假设当前存在两个变量X、Y,通过下列公式计算X、Y之间的皮尔逊相关系数:

% 皮尔逊相关也称为积差相关(或积矩相关)是英国统计学家皮尔逊于20世纪提出的一种计算线性相关的方法。

% 假设有两个变量X、Y,那么两变量间的皮尔逊相关系数可通过以下公式计算:
\begin{equation}
  \rho_{X,Y} = \frac{cov(X,Y)}{\rho_X\rho_Y}=\frac{E(XY)-E(X)E(Y)}{\sqrt{E(X^2)-E^2(X)}\sqrt{E(Y^2)-E^2(Y)}}
\end{equation}
其取值范围为[-1,1],$\rho_{X,Y}=1$时,说明X和Y完全正相关,$\rho_{X,Y}=-1$时,说明X和Y完全负相关,$\rho_{X,Y}$接近0时,说明X和Y无线性相关性。


\section{循环神经网络模型}
经过数十年的发展,神经网络有非常多的种类,按照网络中是否包含循环可以将神经网络分为前馈神经网络和循环神经网络。
\begin{enumerate}
    \item 前馈神经网络:前馈神经网络是一种单元之间连接不形成循环的神经网络。在这种网络中,信息从输入到输出正向流动。前馈神经网络如果只有一层输入节点、一层输出节点,不包含隐藏层,那么被称为单层感知器(Single Layer Perceptron)。若网络由多层计算单元组成,以前馈方式相互连接,则被称为多层感知器(Multi Layer Perceptron)。


\item  循环神经网络
在循环神经网络(RNN)中,单元之间的连接形成了一个定向循环(它们向前传播数据,同时也向后传播数据,从较后的处理阶段到较早的阶段)。这使得它能够表现出动态的时间行为。与前馈神经网络不同,RNNs可以利用其内部存储器处理任意输入序列。这使得它们适用于未分割、连接的手写识别、语音识别和其他一般序列处理器等任务。
\end{enumerate}

假设一个神经元接收$𝐷$ 个输入$x_1,x_2,...,x_D$令向量$x=[x_1;x_2;...;x_D]$来
表示这组输入, 净输入也叫净活性值
(Net Activation)。
并用净输入(Net Input)$z\in \mathbb{R}$表示一个神经元所获得的输入信
号$x$的加权和,

\begin{equation}
\begin{aligned}    
    z &= \sum_{d=1}^D w_d x_d + b \\
      &= w^Tx + b        
\end{aligned}    
\end{equation}

其中$w=[w_1;w_2;...;w_D] \in \mathbb{R}^D$ 是$𝐷$ 维的权重向量,$b\in \mathbb{R}$是偏置。


净输入$𝑧$在经过一个非线性函数$f(\cdot)$后,得到神经元的活性值(Activation)。
\begin{equation}
    a = f(z)
\end{equation}
其中非线性函数$f(\cdot)$称为激活函数。

一个人工神经元的结构如图~\ref{fig:神经元}所示:
\begin{figure}
    \centering
    \includegraphics[width=0.6\linewidth]{人工神经元.png}
    \caption{人工神经元模型}
    \label{fig:神经元}
  \end{figure}

在图~\ref{fig:神经元}中,$\vec{x}$为输入向量,$w$和$b$分别是权重和偏移。

神经网络主要由以下几部分组成:
\begin{itemize}
    \item 输入节点(输入层)。在这一层中不进行任何计算,它们只是将信息传递给下一层(大部分时间是隐藏层)。
    \item 隐藏节点(隐藏层)。中间处理或计算在隐藏层中完成的,然后将输入层的权重(信号或信息)传递给下一层(另一个隐藏层或输出层)。一个神经网络也可以不包含隐藏层。
    \item 输出节点(输出层)。此层位于神经网络的最末层,负责接管来自前面隐藏层所输入的信息或信号。再通过激活函数,最终将得到合理范围内的理想数值,例如用于分类的softmax 函数。
    \item 连接和权重。神经元之间会有边进行连接,每条边会有一定的权重。即每个连接将神经元$i$的输出传递给神经元$j$的输入,每个连接被赋予一个权重$W_{ij}$。
    \item 激活函数。这个函数的作用在于将非线性特征引入到神经网络当中。同时它会将值的范围紧缩至更小,所以一个Sigmoid 激活函数的值区间为[0,1]。深度学习中有很多激活函数,如Sigmoid、Tanh、ReLU 、Softplus、Softmax 等。表\ref{table:激活函数}为常见的激活函数。
    \begin{table}[]
        \caption{激活函数}
        \label{table:激活函数}
        \centering
        \begin{tabular}{|l|l|l|}
        \hline
        名称&表达式&导数\\ \hline
        Sigmoid &  $f(x) = \frac{1}{1+e^{-x}}$ & $f'(x) = f(x)(1-f(x))$
        \\ \hline
        Tanh & $f(x) = \frac{2}{1+e^{-2x}} - 1$ & $f'(x) = 1 - f(x)^2$ \\ \hline
        ReLU & $f(x) = max(0, x)$ & $f'(x)=\begin{cases}
        0& \text{x<0}\\
        1& \text{x>=0}
        \end{cases}$ \\ \hline
        Softplus & $f(x) = log(1+e^x)$ & $f'(x) = \frac{e^x}{1+e^x}$ \\ \hline
        Softmax & $S_i = \frac{e_i}{\sum_j e_j}$ & \\ \hline
        \end{tabular}
        \end{table}
    \item 学习规则。利用学习规则修改神经网络中的权重和阈值。
\end{itemize}

\subsection{RNN}

下图~\ref{fig:循环神经网络}是循环神经网络的示意图。
\begin{figure}
    \centering
    \includegraphics[width=0.6\linewidth]{循环神经网络.jpg}
    \caption{循环神经网络}
    \label{fig:循环神经网络}
  \end{figure}
该图显示了一个循环神经网络被展开成一个完整的神经网络。例如,如果输入序列是时间窗口为T的一组向量,那么网络会被展开成T层的神经网络。
  用公式表示如下:
  \begin{equation}
      \begin{aligned}
          O_t &= g(V\cdot S_t) \\
          S_t &= f(U\cdot X_t + W\cdot S_{t-1})
      \end{aligned}
  \end{equation}

  $x_t$表示第$t$步的输入,例如$x_1$表示时刻1的特征向量。$s_t$表示第t步隐藏层状态,也就是网络中的“记忆”。获得$s_t$的过程是计算位于之前的隐藏层状态及当前层的输入向量。函数$f(\cdot)$通常是一个非线性函数,如tanh或者ReLU。

RNN伪代码如算法\ref{RNN伪代码} 所示。
  \begin{algorithm}[!h]
    \caption{\emph{RNN伪代码}}
    \label{RNN伪代码}
    \begin{algorithmic}[1]
      \Require t时刻的特征向量
      \Ensure t+T时刻的特征向量
      \State 初始化t时刻单元状态
      \For{t $\leftarrow$ $1$ to $T$}
        \State $output_t = activation(input_t, state_t)$
        \State $state_t$ = $output_t$
      \EndFor
    \end{algorithmic}
  \end{algorithm}

\subsection{LSTM}
普通RNN有不能处理长依赖的问题,因此Hochreiter提出了一种长短期记忆网络-LSTM,LSTM是一种特殊的RNN,适用于学习长期依赖。现在LSTM已经被广泛应用于各个领域。LSTM和RNN类似,网络中都具有链式结构,但是LSTM中的循环单元与RNN中简单的$tanh$层不同,其构建了一些“门”(Gate)。利用构建出的“门”单元,能够将历史信息保留在当前节点状态,具体来说,是保留那些权重高的单元删除权重低的单元。
% 让神经网络记住很久之前的重要信息或者忘记最近的不重要信息。

% 所有RNN都具有链式形式。在普通的RNN中,这种循环是一种非常简单的结构,比如简单的tanh层。
\begin{figure}
    \centering
    \includegraphics[width=0.6\linewidth]{普通RNN结构.jpg}
    \caption{普通RNN结构}
    \label{fig:普通RNN结构}
  \end{figure}

% LSTM也具有这种链式结构,但循环单元里面不再是只有单一的神经网络层,而是构建了一些“门”(Gate)。原来的 RNN,由于这种链式结构的限制,很长的时刻以前的输入对现在的网络影响非常小,后向传播时那些梯度也很难影响很早以前的输入,即会出现梯度消失的问题。而 LSTM 通过构建“门”,让网络能记住那些非常重要的信息,比如遗忘门,来选择性清空过去的记忆和更新较新的信息。
\begin{figure}
    \centering
    \includegraphics[width=0.6\linewidth]{LSTM结构.jpg}
    \caption{LSTM结构}
    \label{fig:LSTM结构}
  \end{figure}
  LSTM 首先通过$\sigma$层的“遗忘门”从单元状态中丢弃不重要的信息。遗忘门会读取上一时刻的输出$h_{t-1}$和当前时刻的输入$x_t$,计算出一个维度为n的向量$f_t$,该向量的值均在$[0, 1]$之间。1 表示“完全保留”上个神经元的状态信息,0 表示“完全舍弃”。

\begin{equation}
    \begin{aligned}
        f_t = \sigma(W_f\cdot x_t + U_f\cdot h_{t-1} + b_f)
    \end{aligned}
\end{equation}

下一步是确定该神经元的哪些新状态信息被存放在单元状态中。这里包含两个部分。第一,sigmoid 层,即 “输入门层” ,决定LSTM单元将更新哪些值。然后, tanh 层创建一个新的候选值$z_t$的向量,该向量可以加入到下一层单元状态中。

\begin{equation}
    \begin{aligned}
        i_t = \sigma(W_i\cdot[h_{t-1},x_t] + b_i)
    \end{aligned}
\end{equation}

\begin{equation}
    \begin{aligned}
        \widetilde {C_t} = tanh(W_C\cdot[h_{t-1},x_t]+b_C)
    \end{aligned}
\end{equation}
最后一步是将旧单元状态$c_{t-1}$更新为新状态$c_t$。把旧状态与遗忘门$f_t$相乘,丢弃掉之前无需保留的信息,接着与新状态进行相加,综合得出该神经元的输出的状态,也即更新单元的状态。
\begin{equation}
    \begin{aligned}
        C_t = f_t * C_{t-1} + i_t * \widetilde{C_t}
    \end{aligned}
\end{equation}


\begin{algorithm}[!h]
  \caption{\emph{LSTM伪代码}}
  \begin{algorithmic}[1]
      \Require 一组按时间排列的向量组
      \Ensure 按时间排列的向量组
      \State $\vec C_0 = \vec 0$
      \State $\vec h_0 = \vec 0$
      \For{$t$ \leftarrow $1$ to $T$}
      \State $output_t = activation(dot(W, input_t) + dot(U, state_t) + b)$
      \State $state_t = output_t$
      \State $i_t = activation(dot(state_t,))$
      % output_t = activation(dot(state_t, Uo) + dot(input_t, Wo) + dot(C_t, Vo) + bo)
      % #输入门
      % i_t = activation(dot(state_t, Ui) + dot(input_t, Wi) + bi)
      % 遗忘门
      % f_t = activation(dot(state_t, Uf) + dot(input_t, Wf) + bf)
      % #候选记忆单元
      % k_t = activation(dot(state_t, Uk) + dot(input_t, Wk) + bk)
      
      % c_t+1 = i_t * k_t + c_t * f_t

      \EndFor
  \end{algorithmic}
\end{algorithm}


\subsection{GRU}
门控循环单元(Gated Recurrent Unit,GRU)简化了LSTM模型,不仅合并了遗忘门和输入门,也合并了单元状态和隐藏层状态,在保证训练效果的同时大大减少了参数数量。
% 循环门单元(Gated Recurrent Unit,GRU),由 Cho, et al. (2014)提出。它组合了遗忘门和输入门到一个单独的“更新门”中。它也合并了cell state和hidden state,并且做了一些其他的改变。结果模型比标准LSTM模型更简单,
\begin{figure}
    \centering
    \includegraphics[width=0.6\linewidth]{GRU结构.png}
    \caption{GRU结构}
    \label{fig:GRU结构}
  \end{figure}

  \begin{equation}
    \begin{aligned}
        z_t = \sigma(W_z\cdot [h_{t-1},x_t])
    \end{aligned}
\end{equation}

\begin{equation}
    \begin{aligned}
        r_t = \sigma(W_r\cdot[h_{t-1},x_t])
    \end{aligned}
\end{equation}

\begin{equation}
    \begin{aligned}
        \widetilde {h_t} = tanh(W\cdot[r_t * h_{t-1}, x_t])
    \end{aligned}
\end{equation}

\begin{equation}
    \begin{aligned}
        h_t = (1- z_t) * h_{t-1} + z_t * \widetilde{h_t}
    \end{aligned}
\end{equation}
由图中结构可以看出,GRU是通过一个循环神经网络和“门”机制来不断更新内部参数。



\section{基于关系图结构的RNN}
由前两节分析可知,LSTM和GRU作为两种循环神经网络,都可以很好地提取时序相关性。但是在复杂的清华大学校园网流量环境下,仍然有很多改进空间。根据第3章的实验结果,LSTM和GRU在CAMPUS数据集下准确率仅为65.4\%和67.8\%。经过特征分析,我们发现不同时刻下特征之间的相关性会发生变化。因此本文中我们利用循环神经网络(RNN)对时间依赖性进行建模的同时对特征关系也进行建模。值得指出的是,本文中使用的RNN子类是“门控循环单元”(GRU),相比于普通RNN,它可以很好地捕捉到时序数据中相隔较远的依赖关系。在GRU的矩阵乘法中,我们加入了前文提到的特征关系图(Feature Graph)。

定义无向权重图$G=(V,E,W)$,其中$V$表示特征节点的集合,$|V|=N$; $E$表示特征间的关联关系,即图中的边;$W \in R[N*N]$为特征节点的相似度的加权邻接矩阵,我们称$W$为特征关系矩阵。将时刻t观察到的流量特征向量表示为$X^t$。那么流量预测目的就是:给定$G$下,学得一个函数将$T$个历史图信号映射到未来$T$时刻的图信号:
\begin{equation}
    h[X^{t-T+1}, X^{t-T+2},...,X^{t}; G] \Rightarrow [X^{t+1}, X^{t+2}, ..., X^{t+T}]
\end{equation}

\begin{equation}
    r^{(t)} = \sigma(\Theta_r\star G[X^{(t)},H^{t-1}] + b_r)
\end{equation}

\begin{equation}
    u^{(t)} = \sigma(\Theta_u\star G[X^{(t)},H^{t-1}] + b_u)
\end{equation}

\begin{equation}
    C^{(t)} = tanh(\Theta_C\star_G[X^{(t)},(r^{(t)}\odot H^{(t-1)})] + b_c)
\end{equation}

\begin{equation}
    H^{(t)} = u^{(t)}\odot H^{(t-1)} + (1 - u^{(t)}) \odot C^{(t)}
\end{equation}

其中$X^{(t)}$, $H^{(t)}$表示在时间 $t$ 的输入和输出,$r^{(t)}$ $u^{(t)}$分别是在时间 $t$的复位门和更新门。$\star G$  表示扩散卷积,并且  是对应滤波器的参数。与GRU相似,该模型可用于构建递归神经网络层,并使用反向传播进行训练。
FG-RNN算法的原理如图\ref{fig:FG-RNN原理图}所示。
\begin{figure}[H]
  \centering
  \includegraphics[width=1\linewidth]{FG-RNN原理图.png}
  \caption{FG-RNN原理图}
  \label{fig:FG-RNN原理图}
\end{figure}
FG-RNN算法的伪代码如算法\ref{FG-RNN伪代码}所示。
\begin{algorithm}[!h]
  \caption{\emph{FG-RNN伪代码}}
  \label{FG-RNN伪代码}
  \begin{algorithmic}[1]
    \Require
      一组按时间排序的向量组$\vec{x_1}$,$\vec{x_2}$,...,$\vec{x_T}$, 特征间关系图$G$以及它的邻接矩阵$W$

    \Ensure
      按时间排序的向量组$\vec{h_1}$,$\vec{h_2}$,...,$\vec{h_T}$
    \State 通过Xavier方法初始化$q_{net1}$, $q_{net2}$ and $p_{net}$的所有参数
    \While {$\mathcal{L}$没有收敛}
        \State 利用公式 计算每一个节点$v$的$q_{net1}$, $q_{net2}$ and $p_{net}$
        \For{i $\leftarrow$ $1$ to $T$}
        \State 更新特征关系矩阵$W(e)$的系数
        \State 更新GRU中的参数$W(X)$
        \State 计算$tanh(W(e) + W(X) + b)$        
        \EndFor
        \State 用公式计算目标函数$\mathcal{L}(\theta, \phi)$ 
        \State 用梯度下降法更新$q_{net1}$,$q_{net2}$和$p_{net}$的所有参数
    \EndWhile
    %\State \Return $P_{net}$
  \end{algorithmic}
\end{algorithm}

通常,计算卷积会很费时。 但是,如果$G$很稀疏,则可以使用总时间复杂度 $O(K \mid \varepsilon \mid) \ll O(N^ 2)$ 递归稀疏矩阵乘法来有效地进行计算,本文中计算特征关系图和GRU都可以使用该矩阵乘法。


% 循环神经网络解决了这个问题。它们是带有循环的网络,允许信息持续存在。循环神经网络可以被认为是同一个网络的多个副本,每个副本都会向后继者传递信息。
% \section{实验方案设计及实验流程}
% \begin{equation*} hl=[hl_{1},\ hl_{2},\ \ldots,\ hl_{n-f+1}] \tag{-} \end{equation*}

\section{在RNN中引入特征图信息}
在RNN的训练过程中,加入有效的额外辅助信息往往能够提升训练效果,例如在机器翻译领域,加入文章的关键词、摘要、作者信息等。通常加入信息的方式有以下几种。

\begin{enumerate}
  \item 直接将额外信息向量与当前特征向量叠加。原向量为$p=(p_1,p_2,...,p_n)$,额外信息向量为$e=(e_1,e_2,...,e_n)$,最终输入向量为$w=(p_1+e_1,p_2+e_2,...,p_n+e_n)$。这种方法需要保证额外信息向量与原向量维度相同。
  \item 将额外信息向量与当前特征向量拼接。原向量为$p=(p_1,p_2,...,p_n)$,额外信息向量为$e=(e_1,e_2,...,e_m)$,最终输入向量为$w=(p_1,p_2,...,p_n,e_1,e_2,...,e_m)$。也就是增加输入的维度,缺点是通常要求额外信息的特征与原特征类型保持一致,例如均为词向量。
  \item 增加一个额外的隐藏层,分别使用不同的矩阵进行变化,将结果用$tanh$函数映射到所需的维度。相当于增加一个普通循环神经网络模型和额外信息模型的感知器,然后加载到输出层上。即:
  \begin{equation}
    h'_t= tanh(W(p_t) + W(e) + b^{h'_t})
  \end{equation}
\end{enumerate}
因此,本文采用第三种方式将额外的特征关系图信息与原有特征结合起来。

\section{实验评估}
为了验证本文提出的FG-RNN模型的性能,本文分别在UNSW-NB15、CICIDS2017、CAMPUS三个数据集上,从多个评价维度对FG-RNN进行了实验评估,并与LSTM模型进行了对比。
% \subsection{评估数据集}
\subsection{模型训练}
首先,将数据集按照70\%、15\%、15\%的比例分为训练集、验证集和测试集,分别用于训练模型、调整模型超参数和测试模型。模型的损失函数为交叉熵损失函数,其公式为:
\begin{equation}
  a = \frac{b}{c}
\end{equation}
\subsection{模型稳定性分析}
对于本文提出的FG-RNN模型和所有基准模型,我们分别进行了多次实验。

\subsection{模型准确性分析}
为了验证FG-RNN模型的有效性,本节首先在清华大学无线校园网真实场景下的数据集上开展有效性评估,随后在异常检测领域下的两个经典的公开数据集上进行实验,分别对比了FG-RNN和多个模型的评估效果,最后分析了模型参数的敏感性。


\subsection{训练过程对比}
图\ref{fig:随着训练轮次的增加准确率的变化}中横坐标为FG-RNN和LSTM两个算法随训练轮次的增加,准确率的变化,可以看出LSTM训练时间更短,大约25轮即可稳定,而FG-RNN需要训练50轮以上。

\begin{figure}[H]
    \centering
    \includegraphics[width=1\linewidth]{accuracy_epoches.png}
    \caption{随着训练轮次的增加准确率的变化}
    \label{fig:随着训练轮次的增加准确率的变化}
  \end{figure}

图\ref{fig:loss收敛对比}表示FG-RNN和LSTM两个算法loss收敛的对比,在达到稳定的准确率后,loss也会维持一个相对稳定的水平。
  \begin{figure}[H]
    \centering
    \includegraphics[width=1\linewidth]{loss_epoch.png}
    \caption{loss收敛对比}
    \label{fig:loss收敛对比}
  \end{figure}

\subsection{实验结果对比}

根据第3章的实验结果,LSTM和GRU在CAMPUS数据集下准确率仅为65.4\%和67.8\%,而加入固定的特征关系图后,LSTM准确率即可达到81.2\%,将特征关系图与RNN同时训练,动态更新特征关系矩阵的参数,则FG-RNN算法的准确率可达到85.3\%。



% \subsection{基于真实数据的检测结果}
% 交叉熵损失函数经常用于分类问题中,特别是在神经网络做分类问题时,也经常使用交叉熵作为损失函数,此外,由于交叉熵涉及到计算每个类别的概率,所以交叉熵几乎每次都和sigmoid(或softmax)函数一起出现。

% 我们用神经网络最后一层输出的情况,来观察整个模型预测、获得损失和学习的流程:

% 神经网络最后一层得到每个类别的得分scores;
% 该得分经过sigmoid(或softmax)函数获得概率输出;
% 模型预测的类别概率输出与真实类别的one hot形式进行交叉熵损失函数的计算。
% \begin{figure}
%   \centering
%   \includegraphics[width=0.6\linewidth]{example-image-a.pdf}
%   \caption{Example figure in appendix}
%   \label{fig:appendix-survey-figure}
% \end{figure}




