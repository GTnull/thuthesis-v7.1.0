% !TeX root = ../thuthesis-example.tex

% 中英文摘要和关键字

%TODO: 摘要不要写这么细,异常检测的目的和校园网特点。 BGP劫持,修改掉形容词。摘要不要写太细,背景不要太详细,把贡献说清楚就可以。重新写一遍摘要。扫描流量、攻击流量占比多。具体多多少
\begin{abstract}
  现如今互联网已经成为最重要的信息化基础设施,智能运维作为新兴的网络管理技术倍受关注。 网络流量异常检测是智能运维的关键环节,具有重要意义。然而,现有的流量异常检测研究多是面向仿真的开源数据集环境,少有针对真实网络环境的有效算法。本文面向清华大学校园网真实环境,研究网络流量异常检测算法和系统。校园网流量具有流量规模大、应用类型多、异常种类多且实时变化等特点。现有的异常检测算法在校园网环境下具有以下几个问题:(1)应用类型多导致流量特征复杂,因此很难达到较高的检测率和较低的误报率;(2)不管是传统机器学习算法还是拟合能力更强的深度学习模型,都很难有效地从复杂的流量环境中学习到正常流量的基线;(3)面对海量数据规模时,现有的算法模型无法或很难做到实时性。
  为应对这些问题,本文主要的研究内容和创新点如下:
  % 从网络故障管理的角度来说,做好异常检测可以提前预测故障的发生;从性能管理的角度来说,可以发现性能不佳的区域,避免因误配置、架构不合理导致性能下降;从安全管理的角度来说,可以在网络攻击的前期阶段,及时发现并预警后续攻击,进而做出防御措施。然而,日益复杂的网络环境给异常检测算法带来了诸多问题和挑战。本文以校园网为例,进行网络流量异常检测算法和系统的研究。主要的研究内容和贡献如下:
\begin{enumerate}
  \item 首先在经典开源数据集和清华大学校园网数据集上对多个算法进行了实验评估,分析了现有算法存在的问题。然后设计了一种基于特征关系图的循环神经网络算法(FG-RNN)。通过在RNN算法中引入特征图信息,FG-RNN模型可以增强对特征的利用。实验结果表明,FG-RNN算法检测结果优于现有的基于机器学习、深度学习的算法,且这些提升效果来源于本文引入的特征关系图。
  % 原有的异常检测算法通常直接利用提取的特征进行训练,本文提出的FG-RNN算法有效利用了特征之间的相关性信息,将其加入到神经网络的训练过程中。经过在开源数据集、真实数据集下分别进行的实验验证,本文提出的算法检测结果优于现有的基于机器学习、深度学习的算法。
  \item 在FG-RNN的基础上,设计和实现了基于Spark Streaming的实时异常检测系统,该系统可以在校园网海量的数据下实时异常检测。
  \item 该系统在校园网中检测出的端口扫描进行了定量分析,发现其不仅是安全问题,同时会对无线网性能产生影响。因此进一步提出了两种优化策略,可以有效地缓解该问题。
  % \item 基于该实时异常检测系统,在大规模真实数据集上对FG-RNN和现有神经网络算法进行实验对比,FG-RNN算法效果均优于其他算法,并且这些提升都来自于本文引入的特征关系图。
\end{enumerate}
  

% 现有问题:1.大多数现有算法基于开源数据集,很难应用到真实网络环境中;2.真实网络环境复杂,流量异常种类多,难以找到基线;
% 本文主要工作如下:
% 1.综述了
% 2.提出了一种
% 3.实现了系统,



  % 关键词用“英文逗号”分隔,输出时会自动处理为正确的分隔符
  \thusetup{
    keywords = {大规模校园网, 异常检测, 系统设计, 神经网络},
  }
\end{abstract}

\begin{abstract*}
  Nowadays, the Internet has become one of the most important information infrastructure, as an emerging network management technology, AIOps has attracted much attention. Network traffic anomaly detection is a key component of AIOps. However, most of the existing traffic anomaly detection studies are developed based on simulated environments with open source dataset, and few effective algorithms are available for real network environments. In this paper, we study network traffic anomaly detection algorithm and system for campus networks, in particular, the campus network of Tsinghua University. Campus network traffic is characterized by large traffic volume, many types of application, anomalies and real-time changes. Existing anomaly detection algorithms have several problems in such an environment: (1) multiple application types lead to complex traffic characteristics, so it is difficult to achieve a high detection rate and a low false alarm rate; (2) neither traditional machine learning algorithms nor deep learning models with better fitting ability can effectively learn the baseline of normal traffic from the complex traffic environment; (3) when facing the massive data scale, existing algorithmic models are unable or difficult to work in a real-time fasion. In order to solve the above problems and challenges, the main work and contributions of this paper are as follows.

  \begin{enumerate}
    \item We first evaluate multiple algorithms with both  classic open source datasets and traffic collected from Tsinghua University campus network dataset, and analyzed the problems of these algorithms. Then we design a feature graph based recurrent neural network algorithm (FG-RNN). By introducing feature graph into the RNN algorithm, the FG-RNN model can make a better use of features. The experimental results show that FG-RNN algorithm works better than those algorithms that are based on machine learning and deep learning, and the improvement effects comes from the feature graph we have introduced in FG-RNN.
    \item Based on FG-RNN, we design and implement a real-time anomaly detection system on top of Spark Streaming. This system can detect real-time anomaly from the massive data of our campus network.
    \item We conduct a quantitative analysis on the port scanning activities detected by our system. We find that these activities are not only security problem, but also affect the performance of the wireless network. We further propose two optimization strategies to effectively alleviate this problem.
  \end{enumerate}



  % Use comma as seperator when inputting
  \thusetup{
    keywords* = {Large-scale Campus Network, Anomaly Detection, System Design, Neural Network},
  }
\end{abstract*}
