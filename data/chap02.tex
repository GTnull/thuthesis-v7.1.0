% !TeX root = ../thuthesis-example.tex

\chapter{相关工作综述}

\section{引言}

本章对网络流量异常检测领域的相关工作进行综述。首先介绍校园网网络流量的特点,

其基础是建立一个代表正常/预期网络行为的基线轮廓,并将任何观察到的与该轮廓相比的当前活动的偏差视为异常。这个基线主要通过统计和历史网络流量数据生成。


异常检测是一个重要的领域,自1980年以来,国内外已经有无数学者在这方面做研究。分类、统计、信息理论和聚类。

\citet{ahmed2016survey}将异常检测技术分为分类、统计、信息理论和聚类四类。

异常检测的通用框架如图~\ref{fig:scheme}所示。

\begin{figure}
    \centering
    \includegraphics[width=0.6\linewidth]{异常检测技术的通用架构.png}
    \caption{异常检测的通用框架}
    \label{fig:scheme}
  \end{figure}

本章还讨论了用于网络入侵检测的数据集的研究挑战。



\section{网络流量异常的定义和分类}
Hawkins(1980)给出了异常的本质性的定义\cite{hawkins1980identification}:异常是在数据集中与众不同的数据,使人怀疑这些数据并非随机偏差,而是产生于完全不同的机制。例如在道路交通领域,某条道路的车流量突然增多甚至堵塞或者突然减少,此时车流量数据就是一个异常。因此网络行为的异常就是指那些偏离了正常的、标准的或预期的行为。为了检测网络异常,网络所有者必须有一个预期或正常行为的概念,我们称其为基线。要检测网络行为的异常,就需要持续监控网络中的意外趋势或事件,那些可能改变网络流量特征或者监控指标的恶意行为。


因此我们只关注引起网络流量特征变化的恶意行为,而对于系统权限提升,缓冲区溢出等黑客攻击手段不做研究。
网络异常根据

网络流量异常具体有哪些类别,学术界没有统一的意见。本文中关注的网络流量异常按照产生意图分为恶意和非恶意两类,其中恶意行为主要有拒绝服务攻击,网络扫描,BGP劫持,网络蠕虫,僵尸网络等;非恶意行为主要有物理故障,突发事件等。接下来我们对这些异常分别进行介绍:

\begin{enumerate}
    \item 拒绝服务攻击:拒绝服务(DoS)攻击是一种网络攻击,恶意行为者的目的是通过中断计算机或其他设备的正常运作,使其目标用户无法使用该设备。DoS攻击的功能通常是通过构造大量请求淹没目标机器,直到正常的流量无法处理,导致额外用户的拒绝服务。分布式拒绝服务(DDoS)攻击是一种来自许多分布式来源的DoS攻击,如僵尸网络DDoS攻击。因其攻击成本低、攻击效果明显等特点,DDoS攻击仍然是互联网用户面临的最常见、影响较大的网络安全威胁之一。DoS攻击通常分为2类。(1)缓冲区溢出攻击:一种攻击类型,内存缓冲区溢出会导致机器消耗所有可用的硬盘空间、内存或CPU时间。这种形式的利用通常会导致行为迟缓、系统崩溃或其他有害的服务器行为,导致拒绝服务。(2)洪范攻击:通过用大量的数据包使目标服务器饱和,恶意行为者能够使服务器容量过饱和,导致拒绝服务。为了使大多数DoS泛滥攻击成功,恶意行为者必须拥有比目标更多的可用带宽。

    \item 网络扫描。网络扫描黑客在进行网络攻击之前,首先要进行网络扫描,从中寻找可以攻击的目标。具体来说,黑客需要确定网络中哪些主机是活动的,活动主机上运行了哪些存在漏洞的服务等,从而决定下一步的攻击计划。网络扫描分为主机扫描和端口扫描两种。

    主机扫描主机扫描的目的是确定网络中存在哪些在线的主机或设备。
    端口扫描端口扫描的目的是确定活动主机上开放了哪些端口,运行了哪些网络服务。按照扫描方式来分,端口扫描分为垂直扫描和水平扫描。垂直扫描会扫描某个主机的所有端口,而水平扫描则扫描网络中所有主机的某个固定端口。

    \item BGP劫持。BGP劫持是指攻击者恶意地对互联网流量进行重新路由。攻击者通过谎称拥有一组IP地址(称为IP前缀)的所有权来实现这一目的,而实际上他们并不拥有、控制或路由。BGP劫持就像有人把高速公路上的所有标志都换掉,把汽车流量改道到错误的出口。
    
    \item 网络蠕虫。网络蠕虫不同于计算机病毒。与计算机病毒不同的是,计算机蠕虫不需要附在别的程序内,可能不用用户介入操作也能自我复制或运行。计算机蠕虫未必会直接破坏被感染的系统,却几乎都对网络有害。计算机蠕虫可能会执行垃圾代码以发动分布式拒绝服务攻击,令计算机的执行效率极大程度降低,从而影响计算机的正常使用;可能会损毁或修改目标计算机的文件;亦可能只是浪费带宽。(恶意的)计算机蠕虫可根据其目的分成2类:

    一种是面对大规模计算机使用网络发动拒绝服务的计算机蠕虫,虽说会绑架计算机,但用户可能还可以正常使用,只是会被占用一部分运算、连网能力。
    另一种是针对个人用户的以执行大量垃圾代码的计算机蠕虫。计算机蠕虫多不具有跨平台性,但是在其他平台下,可能会出现其平台特有的非跨平台性的平台版本。第一个被广泛注意的计算机蠕虫名为:“莫里斯蠕虫”,由罗伯特·泰潘·莫里斯编写,于1988年11月2日释出第一个版本。这个计算机蠕虫间接和直接地造成了近1亿美元的损失。这个计算机蠕虫释出之后,引起了各界对计算机蠕虫的广泛关注。

    \item 僵尸网络。僵尸网络(简称 "机器人网络")是指由感染了恶意软件的计算机组成的网络,这些计算机由单一攻击方控制,即所谓的 "僵尸继承者"。在 "机器人携带者 "控制下的每一台机器都被称为 "机器人"。攻击方可以从一个中心点,指挥其僵尸网络上的每一台计算机同时进行协调的犯罪行动。僵尸网络的规模(许多僵尸网络由数百万个僵尸组成)使攻击者能够进行大规模的行动,这在以前的恶意软件中是不可能的。由于僵尸网络一直处于远程攻击者的控制之下,受感染的机器可以接收更新,并在飞行中改变其行为。因此,僵尸牧民往往能够在黑市上租用其僵尸网络部分的访问权,以获取大量经济利益。
    
    \item 物理故障是指路由器故障,链路破坏,断电等不可预测的突发事件。该类异常以两种方式影响着网络链路的流量。
    
    \item 突发事件。突发事件往往是正常的网络操作。
\end{enumerate}



\section{网络流量异常检测算法}


本节将介绍在异常检测领域主流的一些算法,根据所依赖的技术原理的不同,将这些算法分为了基于统计、基于分类、基于聚类、基于信息论、基于深度学习的异常检测算法。

\subsection{基于分类的异常检测算法}

基于分类的技术依赖于专家对网络攻击特征的广泛了解。当网络专家向检测系统提供详细的特征时,具有已知模式的攻击一经发起就能被检测出来。这完全依赖于攻击的签名,作为一个系统,只有当网络专家较早地提供了攻击的签名,它才能够检测出攻击。这说明一个只能够检测到它所知道的系统很容易受到新的攻击,而新的攻击会不断出现不同的版本,并且更加隐蔽地发起。即使创建了新的攻击的签名并将其纳入系统中,最初的损失也是不可替代的,而且修复程序非常昂贵。

基于分类的方法依赖于建立知识库的正常流量活动特征,并将偏离基线特征的活动视为异常活动。其优势在于它们能够检测到完全新颖的攻击,假设它们表现出大量的偏离正常配置文件的情况。此外,由于知识库中未包含的正常流量被认为是攻击,因此会有无意中的误报。因此,异常检测技术需要进行训练,以建立正常的活动配置文件,这很耗时,而且还取决于是否有完全正常的流量数据集。在实践中,获得无攻击的流量实例是非常罕见且昂贵的。此外,在当今׳s动态和不断变化的网络环境中,保持正常配置文件的更新是非常困难的。在现有的大量基于分类的网络异常检测技术中,我们主要讨论以下四种技术。

\citet{2002AEskin} 引入无监督SVM的概念来检测异常事件。常规的SVM的原理是推导出一个超平面,使得正类样本和负类样本之间的分离余量最大化,将特征空间中的两类数据进行分离。标准的SVM算法是一种监督学习方法,需要标记数据来创建分类规则。而该算法经过改进SVM,试图将整个训练数据集从原点分离出来,找到一个以最大余量将数据实例与原点分离的超平面,

\citet{Hu2003Robust} 提出了一种忽略噪声数据的异常检测方法,该方法使用Robust SVM(RSVM)来开发。标准的SVM有一个主要假设,即所有的训练样本数据都是独立且相同分布的(i.i.d)。但是在实际场景中,训练数据往往包含噪声,这就会导致标准SVM会学习出一个高度非线性的决策边界,从而导致通用性较差。基于此,RSVM以类中心的形式加入了平均化技术,使得决策面更加平滑。此外,RSVM另一个优点是能大大降低支持向量的数量,从而减少运行时间,提高效率。

% TODO
\citet{kruegel2003bayesian} 假设异常检测系统包含许多模型,用于分析一个事件的不同特征,指出了在这种系统下异常检测技术造成高误报率的两个主要原因。一是异常检测系统通过将多个概率模型的输出进行汇总,而每个模型往往只给出一个事件的常态/异常的得分或概率,从而导致高误报率;二是异常检测系统无法处理那些不正常但合法的行为,如CPU利用率、内存使用率突然增高等。基于贝叶斯网络的概念,\citet{kruegel2003bayesian}提出了一种解决上述问题的方法。对于一个输入事件的有序流($\symbf{S}=e_1,e_2,e_3...$),异常检测系统决策每个事件是正常还是异常。该决策基于k个模型($\symbf{M}=m_1,m_2,...,m_k$)的输出($o_i|i=1,2,...,k$)和可能的附加信息($\symbf{I}$)。应用贝叶斯网络来识别异常事件,引入根节点,根节点代表一个具有两种状态的变量。一个子节点用于捕捉模型׳的输出,子节点与根节点相连,预计当输入异常或正常时,输出事件会有所不同。贝叶斯网络最近的应用可以在电信网络中找到(Deljac等,2015)。
贝叶斯网络是对包含不确定性的领域进行建模的一种有效方法。一个离散的随机变量用一个有向无环图(DAG)表示,其中每个节点反映了随机变量的状态,并包含一个条件概率表(CPT)。CPT的任务是提供一个节点处于特定状态的概率。在贝叶斯网络中,节点之间存在着父子关系,这表明子节点所代表的变量依赖于父节点所代表的变量。由于这种网络可以用于事件分类方案,因此也适用于网络异常检测。


神经网络已经被应用于各个应用领域,如图像和语音处理,但其对计算量的要求很高。神经网络对数据进行分类的优势也可被用于网络异常检测。在网络异常检测领域,神经网络通常会和其他技术进行结合,如统计方法。2020outlier 提出了一个多层的前馈神经网络,该神经网络可以用来进行异常值的检测。具体来说,Replicator Neural Networks是一个多层前馈的神经网络 (multi-layer feed-forward neural networks),在输入层和输出层之间放置了三个隐藏层。它的目标是通过训练在输出层以最小的误差重现输入数据模式。由于该模型中间隐藏层节点的个数少于输入输出层节点的个数,这样就起到了压缩数据和恢复数据的作用。
RNN的示意图如图~\ref{fig:rnn}所示。

\begin{figure}
    \centering
    \includegraphics[width=0.6\linewidth]{RNN.jpg}
    \caption{Replicator Neural Networks示意图}
    \label{fig:rnn}
  \end{figure}


\citet{2001HIDE} 提出了一种神经网络与统计模型相结合的分层入侵检测系统。将神经网络分类器的输出表示为一个连续变量($t$),其中-1表示有绝对把握的入侵,1表示没有攻击。此外,自组织地图(SOM)被用于网络异常检测。Ramadas等(2003)提出,利用SOM,可以对网络流量进行实时分类。SOM依赖于这样一个假设,即网络攻击可以由不同的神经元组来描述,这些神经元组与其他神经元组相比,在输出神经元图上覆盖更大的区域。\citet{5592568} 开发了一个由反向传播算法训练的前馈神经网络,利用给定的数据集与计算机网络在正常和异常行为期间的相关信息来检测异常。

\subsection{基于统计的异常检测算法}

统计技术的主要挑战是找到减少硬阈值引起的误报产生的方法[12]。例如,可以使用统计信号处理程序来提高检测率,同时减少误报,如Lakhina等人在主成分分析工作中所做的工作[10,69,70]。

小波分析
小波分析的重点是对非稳定数据序列进行建模。这种数据序列可能包含在长时间内振幅和频率都会变化的信号。与使用三角多项式的傅里叶分析不同,数据序列是用小波来建模的,小波是强大的基函数,在时间和频率上是局部的,允许被表示的序列和它们的系数之间有密切的联系。这样一来,小波分析从根本上是一种关于特定数据的细节层次的描述方式,这些数据可以是图像、曲线、曲面等。

Callegari等人[71]提出了一种利用小波与草图相结合的实时异常检测方法。它是通过提取NetFlow轨迹,并将其转化为ASCII数据文件,进行路由器级别的分析。经过格式化后,通过哈希函数将不同的流量汇总到草图表中。接下来,将时间序列提交给小波变换,以达到发现不连续的目的。

另一项使用小波的研究是由Hamdi和Boudriga[72]产生的。它依赖于通过区分危险和非威胁性的异常来识别与攻击相关的异常。这项任务是在周期观测概念的基础上完成的,小波理论被用来分解一维信号,以分析其特殊频率和时间定位。

主成分分析
主成分分析(PCA)是一种广泛应用于计算机网络异常检测的统计技术。它被定义为一种降维方法,由n个相关变量组成的数据集可以映射到一个新的、缩小的k个变量集上,即主成分(PC),其中k≪n。这些PC是一组正交向量,定义了k个子空间,并且是不相关的,其排列方式使第一成分保留了所有原始变量的大部分变化[73,74]。

Lakhina等[10]是这一领域的先驱,他们利用PCA将流量测量结果有效地分离为正常和异常子空间,解决了网络宽流量的异常诊断问题。其主要思想是,PCA的结果是k个变量的缩减集合(主成分或k个子空间),它对应于正常的网络流量行为,而剩余的m个成分的子空间(m=n-k)则由异常或噪声组成。然后,将每一个新的流量测量值投射到这两个子空间上,这样就可以设置不同的阈值来将这些测量值划分为正常或异常。他们的工作是基于PCA的异常检测方法在过去十年中受到大量关注的原因。然而,尽管这是一项声名显赫的工作,并在该领域取得了良好的结果和进步,但它收到了来自不同作者的一些批评,主要与PCA的校准灵敏度有关,如Ringberg等[75]的报道。

Ringberg等[75]等[76,77]对Lakhina等[10,70]关于PCA的研究进行了批评,概述了关于其灵敏度的四大挑战。(i)假阳性率受正常子空间中的小噪声影响;(ii)流量聚集程度可以减轻PCA的价值;(iii)大的异常可以感染正常子空间;(iv)PCA产生的缩小子空间与异常的原始空间源之间没有映射。

这样,Pascoal等人[78]提出的异常检测方法采用了鲁棒PCA检测器与鲁棒特征选择算法合并,以获得对不同网络背景和环境的适应性。此外,这种鲁棒PCA方法不需要完美的地真进行训练,这也是[75]中讨论的标准PCA的局限性之一。在[79]中,作者提出了ADMIRE,它是三步草图和基于熵的PCA的结合,由于PCA的熵时间序列不同,因此可以获得更好的真假阳性率,同时能够捕捉不同类型的异常。此外,O'Reilly等[80]在从有异常的数据集中寻找PC时超越了这些限制,提出了最小体积椭圆PCA(MVE-PCA)方法,该方法包括通过在训练数据集周围创建一个软边最小体积椭圆来解决凸优化问题,从而降低数据中存在的异常的影响。

尽管如此,Camacho等人[81]还是积极地认为,无论是最初的PCA提案还是批判性研究者,都无法有效地超越使用PCA进行异常检测的缺点。为了克服这些缺点,作者采用了基于PCA的多变量统计过程控制(MSPC)方法,该方法同时监控Q统计量和D统计量。因此,有可能建立控制限,以便在持续超过异常情况时发现异常情况。此外,MSPC方法有贡献图用于寻找异常的根本原因。数据预处理依赖于特征即计数器的方法,其中变量是整个给定时间间隔内记录某个事件的次数的计数器。这与Lakhina等人[10]的想法不同,后者将计数器视为简单的定量变量。

协方差矩阵
协方差矩阵是二阶统计,已经被证明是一种强大的异常检测方法。该领域的一个有趣的方向是寻找哪些变量最能标记网络异常,提高检测性能。

[85]中介绍的工作采用协方差矩阵分析来检测洪水攻击。该方法将网络流量建模为协方差矩阵样本,以利用时间顺序样本中包含的统计资产达到检测泛滥攻击的目的。然后直接利用协方差矩阵的变化和相关特征的差异来揭示正常流量和各种类型的泛洪攻击之间的变化。

谢淼[86]通过处理相邻数据段的集合,借助随机变量,利用它们的空间可预测性来判断哪些数据段表现异常,以基于段的方式进行异常检测。该方法采用了按Spearman's rank相关系数和差分压缩概念近似的样本协方差矩阵,以大幅降低计算成本。

Huang等人[87]支持用协方差矩阵代替上一节讨论的传统PCA进行降维。他们指出,静态地选择k个主成分,除了由于对代表法线子空间的维度的微小变化敏感,只允许弱启发式,对实时变化的捕捉能力差。因此,为了克服基于方差的方法的局限性,作者提出了一种基于距离的降维方法进行异常检测。根据其类型的不同,异常点会在观察到的交通量的协方差矩阵中造成不同类型的偏差。这些偏差允许对检测到的异常情况进行分类,并对缓解行动进行即时决策。他们的建议还能够适应测试数据中不断变化的模式,使模型在任何时候都只使用几个重要的维度。



早期的异常检测方法往往基于统计与概率模型,也就是假设-检验的方法。首先对数据的分布做出假设,然后找出假设下所定义的“异常”,因此往往使用极值分析或假设检验。
比如对最简单的一维数据假设服从正态分布,然后将距离均值某个范围以外的点当做异常点。推广到高维后,假设各个维度相互独立。这类方法的好处速度一般比较快,但是因为存在很强的“假设”,效果不一定很好。

% \citet{Nong2010An} 基于卡方检验的距离测量方法,将卡方检验的理论应用于异常检测。根据此技术,首先需要建立一个正常事件的基线,该方法的
入侵检测技术也是利用统计理论发展起来的;例如,Ye和Chen(2001年)在异常检测中使用了成熟的chi-square理论。根据这种技术,建立了一个信息系统中正常事件的档案。这种方法的基本思想是既要检测出与正常事件有较大偏差的异常事件,又要检测出入侵事件。基于chi-square检验统计量的距离测量方法被开发为
\begin{equation}
    \chi^2 = \sum_{i=1}^n \frac{(X_i - E_i)^2}{E_i}
\end{equation}

其中$X_i$为第$i$个变量的观测值,$E_i$为第$i$个变量的期望值,$n$为变量的数量。

当一个变量的观测值接近预期时,$\chi^2$的值就会很低。根据$3\sigma$定律,当观测值的$\chi^2$大于$\bar{X^2}+3S_X^2$时,该值被视为异常。

\citet{Christopher2002Service} 提出了一种用于检测异常网络流量的统计处理单元,更具体地说,是为了检测R2L和U2R等罕见的攻击。开发了一种度量方法,使系统能够自动搜索不同服务请求的相同特征。根据以下三个主要特征计算出请求的异常得分。
请求的类型;
请求的长度;以及
有效载荷分布。
网络管理员定义了一个阈值,以便对异常请求发出警报。异常得分的计算方法如式(5),其中有效负载分布的权重大于其他属性。(5)
基于统计学理论的原理,我们开发了不同类型的技术来检测异常,接下来将讨论。

在时间序列异常检测领域,最常见的基于统计的算法为ARIMA,即差分自回归移动平均模型[8]。我们将流量信号分解为两部分,一是遵循一定规律、可预测的正常变化,二是由突发性变化组成、不可预测的异常情况。ARIMA分析和建模用于网络流量预测,能够检测和识别流量异常或异常值。


\subsection{基于信息论的异常检测算法}

信息论是一门以信息量化和冗余分析为核心的数学学科。它的前身是1948年Claude E. Shannon在寻求信号处理和通信操作的数据压缩、传输和存储时提出的设想[125]。然而,它的应用扩展到许多其他目的,如电信、估计、决策支持系统、模式识别等[126]。有几种信息理论测量方法,如香农熵、广义熵、条件熵、相对熵、信息增益和信息成本等。

其用于异常检测的目的主要是依靠计算指定流量特征的相互信息或熵值,以识别其上的异常分布。由于它采用了交通相关特征的时间序列的统计特性(如高斯),这种方法可能会导致不准确。

熵
熵是最著名的信息理论度量,定义为随机变量的值或随机过程的发生所涉及的等效概率或不确定性。考虑到熵在异常检测领域的应用,它可以有效地将流量特征,如源/目的端口或IP地址,描述为分布,因为有某些类型的异常会对这些分布造成重大干扰。通过这种方式,可以检测到例如由目的端口熵的变化表示的端口扫描攻击,甚至由源/目的IP地址熵的变化表示的DDoS攻击的发生[127]。

David等人[128]提出了一种通过快速熵方法和使用基于流量的分析来增强对DDoS攻击的检测。作者将观察到的流量汇总成一个单一的流量,并考虑到每个连接在一定时间间隔内的流量数,而不是取每个连接的数据包数。第二步基本上是计算每个连接的流量计数的快速熵。最后,根据快速熵和流量计数的均值和标准差生成一个自适应阈值。阈值随流量模式状况不断更新,提高了检测精度,同时快速熵的使用减少了计算处理时间。

Amaral等[129]为了对网络流量进行深度检测,提出了一种基于特征的异常检测系统,该系统同时使用IP Flow属性和图表示。该检测基于Tasallis熵,是Shannon熵的泛化。主要的分歧点在于它有一个参数来定义哪些概率会对熵的结果有贡献。它调整了异常检测器的敏感性,使其能够适应不同类型的网络,并检测到比基于量分析的方法检测到的攻击更多的非表达性攻击。

Bhuyan等人[130]提出的工作带来了一种基于离群值的异常检测方法,使用广义熵和互信息来创建一种能够选择相关的、非冗余的特征子集的特征选择技术。作者认为,由于相互信息降低了一个随机变量的不确定性,而广义熵衡量了数据中的不确定性量,因此他们使检测速度更快,更准确。

此外,Berezinski等人[131]为了检测现代僵尸网络恶意软件,引入了一种基于香农熵的网络异常检测器。他们的方法创建了一个网络配置文件,它存储了5分钟滑动时间窗口中的最小和最大熵值。这些值被用于与观察到的熵进行比较。这定义了一个阈值,因此,可以识别不同特征分布的异常分散或集中。最后,作者使用流行的分类器,如决策树和贝叶斯网络,以便对异常进行分类。

Behal和Kumar[132]指出,由于DDoS攻击和闪电事件会引起网络流量模式的大幅改变,基于信息理论的熵或分叉可以快速捕捉网络流量行为中的这种差异。因此,他们提出了一种广义异常检测算法,该算法利用了流量之间的熵差。他们采用了一组广义的φ-熵和φ-分歧度量,其中检测效率与合法流量和攻击流量之间的信息距离直接相关。所提出的算法对闪光事件和高速率DDoS的检测精度较高,克服了文献中其他信息理论方法的结果。

信息理论的测量方法可以用来建立一个适当的异常检测模型。在Lee和Xiang(2001)的一篇论文中,使用了几种度量方法,如熵、条件熵、相对熵、信息增益和信息成本来解释数据集的特征。我们对这些度量方法的定义如下。

熵是信息论的一个基本概念,它衡量一个数据项集合的不确定性。对于一个数据集D,其中每个数据项都属于一个类(x ),D相对于分类的熵定义为: 1.
(13)
其中P(x)为x在D中的概率。


条件熵是指D的熵,给定Y是概率分布()的熵,为
(14)
其中P(x,y)是x和y的联合概率和x给定y的条件概率。


相对熵是指定义在相同的两个概率分布p(x)和q(x)之间的熵,具体为
(15)

相对条件熵是指定义在同一和上的两个概率分布(和)之间的熵。
(16)

信息增益是对数据集D中某一属性或特征A的信息增益的衡量,是
(17)
其中,值A为A的可能值集,Dv为D的子集,其中A的值为v。


Ambusaidi等人(2014)中提出了一种基于非线性相关系数(NCC)的相似性测量方法,以提取网络流量之间的线性和非线性相关性。提取的相关信息用于检测恶意网络行为。Pearson׳s相关系数是一种基本的线性相关方法,用于找出两个变量之间的依赖关系(Ahmed等,2015c),然而,在一些数据集中,不同变量之间存在非线性相关,如网络流量中。NCC由Wang等(2005)定义,如式(18),其中和为变量X和Y的修正熵。

给定一组m个正常训练数据实例,首先计算NCC。对于任何传入实例,传入实例与正常实例之间的NCC记录为 。对于用户定义的阈值σ,其范围在0和1之间,如果NCC的差异大于σ,则认为一个传入流量实例是异常的(19)。


在Tan等(2014a)中,针对DoS攻击检测,提出了一个利用多元相关分析(MCA)的系统,通过提取网络流量特征之间的几何相关性,来实现网络流量的精确特征分析。检测过程主要包含三个步骤,如图6所示。在步骤1中,在一个明确的时间区间内生成基本特征。第2步包含多元相关分析,应用 "三角区域图生成 "模块,提取第一步得出的每个流量实例中两个不同特征之间的相关性。第三步是基于训练和测试阶段的决策。


基于这些知识,可以建立适当的异常检测模型。有监督的异常检测技术需要先有一个训练数据集,再有一个测试数据来评估模型的性能。在这种情况下,首先,使用信息理论措施来确定模型是否适合测试新数据集。Noble和Cook(2003)在基准DARPA和UNM审计数据集上进行了实验,以证明信息理论措施的效用,并得出结论,它们可以用来创建高效的异常检测模型,也可以用来解释它们的性能

Tan等人(2014a)中的多变量相关分析方法的概念被纳入到网络流量实例的表征中,并将其转换为相应的图像。这些图像被用于DoS攻击检测,基于一个广泛使用的异构度量,即地球移动者距离(Earth Mover׳s Distance,EMD)(Rubner等,1998)。EMD考虑了跨区域匹配,比其他一些著名的异同度测量方法更准确地评估了分布之间的异同度。

\subsection{基于聚类的异常检测算法}

聚类分析的目的是将一组对象分成类似对象的类别。这些类或组称为聚类,其对象彼此相似(以一种或另一种方式),与其他聚类中的对象不同。基于聚类的过程对变化的适应性很强,有助于找出区分不同群体的有用特征。聚类技术可用于离群检测,识别与任何聚类相距太 "远 "的值,或作为其他算法/方法的预处理步骤。此外,分类是区分对象的群组或类的有效资源;然而,它需要收集和标记大量的训练元组或模式,分类器使用这些元组或模式来对每个群组进行建模,这通常是昂贵的[96]。

Rajasegar[97]提出了一种基于分布式超球面集群的算法,用于无线传感器网络中的异常检测。通过将数据向量分类为正常或异常,利用聚类对每个节点的流量数据进行建模。通过使用k个最近邻(KNN)集群的平均集群间距离来识别异常集群。这是在分布式方案下工作的,传感器节点报告集群摘要,在与其他节点通信之前,由中间节点进行合并,从而使通信开销最小化。

Mazel等人[98]介绍了一种非监督的方法来检测和描述网络异常。该方法最初的工作原理是采用聚类技术,将子空间聚类与证据积累聚类和聚类间结果关联相结合,以盲目识别流量中的异常。

K-means是异常检测领域流行的一种聚类技术,能够将数据分为不同的类别。但是,它存在局部收敛性和对聚类中心点选择的敏感性等缺点。因此,许多研究者尝试将k-means与其他技术相结合,以克服这些缺点。Karami和Guerrero-Zapata[99]介绍了一种基于粒子群优化(PSO)和k-means与局部优化混合的模糊异常检测系统,以确定最优的簇数。它分为两个阶段:训练阶段旨在通过将PSO的全局搜索的新颖边界处理方法与k-means的快速收敛相结合,找到近似最优解;从而避免陷入局部最优解的困境。模糊方法用于检测阶段,在检测阶段,通过对入侵活动的可靠检测,降低假阳性率。这是由于任何数据(正常或攻击),都可能与某些集群处于近距离。

Carvalho等人[100]开发了一种主动式网络监控系统,可以检测异常事件,减少决策中的人工干预和错误概率。他们的建议包括创建一个名为DSNSF(使用流量分析的网段数字签名)的网络配置文件,通过对蚁群优化(ACO)元启发式的修改,使用聚类方法描述正常的网络使用情况,称为ACODS。ACODS在集群集的大量高维输入数据中,通过无监督学习机制优化提取行为模式,对网络流量发现进行表征。然后,为了检测异常行为,作者使用了名为动态时间扭曲(DTW)的模式匹配技术。他们首先计算每个时间区间内真实流量与正常曲线的相似度;然后,计算序列之间的距离,并提供基于形式和距离的测量方法。所提出的告警系统采用七种流量属性工作,利用熵来总结IP地址和端口特征的相关信息。当检测到异常时,ACODS会提供一份包含IP流量信息的完整报告,说明每个属性对检测到的异常时间间隔的影响。ACODS具有平方复杂度,导致解的收敛要经过多次迭代,作者试图通过使用局部搜索和信息素更新来缓解。

在,Dromard等[101]提出了基于增量网格聚类算法IDGCA和离散时间滑动窗口的无监督异常检测器ORUNADA。增量网格聚类比通常的聚类算法更有效率,因为后者只更新之前的特征空间分区,而不是每当增加或删除很少的点时,就对整个空间进行重新分区。然后,系统合并这些更新的分区,以努力识别最不相似的异常值。增量网格聚类的使用有助于降低系统复杂度,从而使其在实时检测方面更加可行。

关于SDN及其面临的挑战,如高密度和多种主机,He等人[102]最近开发了一种用于异常检测的两阶段无监督聚类算法。第一阶段是一个特征选择过程,用于去除数据集中不必要的特征。其基础是计算最大信息系数(MIC),它描述了两个连续特征之间的关系,而相关性是离散特征的对称不确定性估计。在选择相关特征后,基于密度峰的聚类算法将缩减后的数据集分为正常模式和错误模式。他们的实验结果证明,当采用典型的SDN控制器层次结构时,可以在每个控制器中对流量数据进行局部分析。这就减少了在网络中洗牌的流量。

异常检测方法的一些主要局限性基本上是:没有标签数据;发现新的未知异常模式;噪声数据;高误报率。为了克服这些问题,Bigdeli等人[103]提出了一种基于增量两层集群的异常检测结构。其核心思想是对网络数据进行聚类,并将这些聚类表示为高斯混合物模型,因此该模型可以对新的实例进行分类,也可以检测并忽略冗余的实例。此外,针对误报率较高的问题,采用集体标记的方法,对新入库实例进行集体标记和增量标记。


聚类指的是无监督学习算法,它不需要预先标记数据来提取相似数据实例的分组规则(Jain等,1999)。虽然有不同类型的聚类技术,但我们讨论常规聚类和共聚类对网络异常检测的有用性。常规聚类和共聚类的区别在于行和列的处理。常规聚类技术如k-means(Ahmed和Naser,2013)考虑数据集的行进行聚类,而共聚类则同时考虑数据集的行和列来产生聚类(Ahmed等人,2015d)。

下面简单讨论一下使用聚类检测异常时总是要做的三个关键假设。
假设1:由于我们只能创建正常数据的聚类,因此,后续任何与现有正常数据聚类不相适应的新数据都被认为是异常数据;例如,由于基于密度的聚类算法不包括聚类内的噪声(Ester等人,1996),噪声被认为是异常数据。

假设2:当一个簇同时包含正常数据和异常数据时,已经发现正常数据靠近最近的簇中心点,但异常数据远离中心点(Ahmed和Naser,2013)。在这种假设下,异常事件使用距离得分来检测。

假设3:在一个具有不同大小的聚类中,较小和较稀疏的聚类可以被认为是异常的,较厚的聚类是正常的。属于大小和/或密度低于阈值的聚类的实例被认为是异常的。


Münz等人(2007)对异常数据采用的方法非常直接。他们使用k-means聚类来生成正常和异常聚类。一旦实现聚类,就使用以下假设进行分析。

如果一个实例比异常簇中心点更接近正常,则该实例被列为正常,反之亦然。


如果实例与中心点之间的距离大于预定义的阈值(dmax),则该实例被视为异常;以及


如果一个实例比正常聚类中心点更接近异常聚类中心点,或者它与正常聚类中心点的距离大于预定义的阈值,则被视为异常。


Petrovic等(2006)提出了一种基于聚类评价技术组合的聚类标签策略。将Davies-Bouldin聚类评价指数和聚类中心直径的比较结合起来,以充分应对攻击向量的特性。他们考虑了相应聚类的紧凑性和它们之间的分离度,以及区分分析网络中 "正常 "和 "异常 "行为的主要参数。然而,他们并没有解释他们的k-means聚类使用k=2的原因。根据他们的方法,攻击向量通常非常相似,如果不是完全相同的话;例如,在大规模攻击的情况下,相应的聚类是非常紧凑的,这种聚类的Davies-Bouldin指数要么是0(当非攻击聚类是空的时候),要么是非常接近0。考虑到攻击向量之间的预期相似性,因为攻击聚类的中心点的直径预期比非攻击聚类的直径小,他们可以区分正常和异常的聚类。

Portnoy等(2001)提出了基于宽度的聚类来对数据实例进行分类。宽度是恒定的,对所有聚类都保持不变。一旦进行聚类,基于正常实例在整个数据集中占压倒性比例的假设,N\%的聚类是正常的,其余是异常的。利用这一假设,Leung和Leckie(2005)提出了一种基于密度和网格的聚类算法,该算法适用于无监督的异常检测。

Syarif等人(2012)研究了各种聚类算法应用于异常检测时的性能。他们使用了五种不同的方法,即k-means、改进的k-means、k-medoids、期望值最大化(EM)聚类和基于距离的异常检测算法。表3展示了用于网络异常检测的聚类算法的性能评价。



聚类算法通常是基于距离/密度发现异常点。基于距离/密度的异常点检测方法的关键步骤在于给每个数据点都分配一个离散度,其主要思想是:
针对给定的数据集,对其中的任意一个数据点,如果在其局部邻域内的点都很密集,那么认为此数据点为正常数据点,而异常点则是距离正常数据点最近邻的点都比较远的数据点。通常有阈值进行界定距离的远近。
异常检测领域下常用的聚类算法有k-means、LOF、孤立森林、高斯混合模型[20]等。

\subsection{基于深度学习的异常检测算法}
随着深度学习的兴起,越来越多的学者尝试用深度学习算法来进行异常检测,尤其是针对时间序列数据,深度学习模型往往表现出惊人的效果。
常用的深度学习算法为变分编码器、神经网络[6][14]、生成对抗网络、LSTM[17]、RNN[3][4][10][12][13][15]等。以变分自动编码器(Variational Auto-Encoder)[5]为例,其利用自编码器的重构误差和局部误差,针对时间序列的异常检测的场景,达到了很好的效果。

\section{异常检测领域开源数据集介绍}
数据集主要由KDDCUP99, CICIDS等。网络流量异常检测领域最为经典的数据集当属KDD99,但是这个数据集年代过于久远,对于现在的网络环境早已不适用。 NSL-KDD是为了解决KDD'99数据集的一些固有问题而提出的数据集,这些问题在[1]中已经提到。虽然,这个新版本的KDD数据集仍然存在McHugh所讨论的一些问题,并且可能不能完美地代表现有的真实网络,但由于缺乏基于网络的IDS的公共数据集,我们相信它仍然可以作为一个有效的基准数据集来帮助研究人员比较不同的入侵检测方法。

此外,NSL-KDD训练集和测试集的记录数量是合理的。这一优势使得在完整的集合上运行实验是经济实惠的,而不需要随机选择一小部分。因此,不同研究工作的评价结果将具有一致性和可比性。

CICIDS2017数据集包含了良性的和最新的常见攻击,与真实的现实世界数据(PCAPs)相似。它还包括使用CICFlowMeter进行网络流量分析的结果,并根据时间戳、源和目的IP、源和目的端口、协议和攻击(CSV文件)对流量进行了标注。同时还提供了提取的特征定义。

生成真实的背景流量是我们构建这个数据集的首要任务。我们使用了我们提出的B-Profile系统(Sharafaldin,等人,2016)来对人类交互的抽象行为进行剖析,并生成自然的良性背景流量。对于这个数据集,我们基于HTTP、HTTPS、FTP、SSH和电子邮件协议建立了25个用户的抽象行为。

数据采集期从2017年7月3日(周一)上午9点开始,到2017年7月7日(周五)下午5点结束,共5天。其中周一为正常日,只包括良性流量。实施的攻击包括蛮力FTP、蛮力SSH、DoS、Heartbleed、Web攻击、渗透、僵尸网络和DDoS。它们在周二、周三、周四和周五的上午和下午都被执行过。

在我们最近的数据集评估框架中(Gharib等人,2016),我们确定了建立一个可靠的基准数据集所必需的11个标准。之前的IDS数据集都无法覆盖这11项标准的全部内容。在下文中,我们简要地概述了这些标准。

THU-IDS 清华校园网数据集,该数据集为真实流量,将于第三章进行介绍。

% https://www.unb.ca/cic/datasets/nsl.html
\section{异常检测算法对比}
对比
不同机器学习方法在NSL-KDD数据集上的效果,
\section{现有异常检测算法存在的问题}
近些年出现了数以千计的异常检测算法,它们的检测原理,适用范围以及所使用的流量特征各不相同。通过上述实验室对比我们可以看出,在应对大规模、应用类型多、异常流量是常态且多种异常相互叠加的场景下,现有异常检测算法大多存在以下缺陷:
1.	面对海量数据规模时,无法或很难做到实时性;
2.	应用类型多导致的流量特征复杂,因此很难达到较高的检测率和较低的误报率;
3.	大多数异常检测算法仅能报告是否发生异常,难以对确定异常种类和定位异常来源给出指导性意见。
由于上述缺陷,当前大多数异常检测算法仍处于概念验证或模拟实现的阶段,距离实际的部署和应用还有不小的距离。
