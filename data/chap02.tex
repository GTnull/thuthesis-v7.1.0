% !TeX root = ../thuthesis-example.tex

\chapter{相关工作综述}

\section{引言}

本章对网络流量异常检测领域的相关工作进行综述。首先介绍了网络流量的特点,


异常检测是一个重要的领域,自1980年以来,国内外已经有无数学者在这方面做研究。分类、统计、信息理论和聚类。

\citet{ahmed2016survey}将异常检测技术分为分类、统计、信息理论和聚类四类。


本章还讨论了用于网络入侵检测的数据集的研究挑战。



\section{网络异常的定义和分类}
Hawkins(1980)给出了异常的本质性的定义\cite{hawkins1980identification}:异常是在数据集中与众不同的数据,使人怀疑这些数据并非随机偏差,而是产生于完全不同的机制。例如在某个季节里,某一天的气温很高或很低,这个温度数据就是一个异常。网络异常是指那些可能改变网络流量特征或统计指标的恶意行为。
网络异常大致可以分为物理故障,网络扫描,BGP前缀劫持,拒绝服务攻击,蠕虫攻击等几种类型。


\section{异常检测}
目前,学术界和工业界已经提出了一系列 KPI 异常检测算法。这些算法可以概括地分成基于窗口的异常检测算法,例如奇异谱变换 (singular spectrum transform) ;基于近似性的异常检测算法 ;基于预测的异常检测算法,例如 Holt-Winters 方法、时序分解方法、线性回归方法、支持向量回归等 ;基于隐式马尔科夫模型的异常检测算法 ;基于分段的异常检测算法 ;基于机器学(集成学习的异常检测算法等类别。

\subsection{基于时间序列的异常检测}
时间序列是将某种统计指标的数值,按时间先后顺序排序所形成的数列。时间序列的预测就是通过分析时间序列,根据时间序列所反映出来的发展过程、方向和趋势,进行类推或延伸,预测下一段时间或以后若干年内可能达到的水平。时间序列的异常检测就是通过历史的数据分析,查看当前的数据是否发生了明显偏离了正常的情况。主要的时间序列模型有移动平均、指数平均等等。
IMC’2015[3]通过有监督的机器学习算法来解决手动和迭代的调整检测器参数和阈值的难题。多年来,人们提出了数十种异常检测器,用于密切监控设备性能及发现异常。但是部署它们是一个巨大的挑战,需要人工手动和迭代的调整检测器参数和阈值。

该文章通过对多个检测器中的性能数据提取异常特征;然后用特征和标签训练随机森林分类器,自动选择适当的参数和阈值。有以下局限性:
1.有监督学习需要带标签的数据;
2.这种算法受限于训练集中的异常,也就是无法判别未来出现的新的异常;
因此,WWW’2018[4]提出了无监督的机器学习算法,即针对周期性KPI数据,使用基于VAE(变分自动编码器)的异常检测算法。

\subsection{基于日志的异常检测}
系统产生的日志是非结构化文本,并且有信息量巨大、类型繁多等特点,这就为分析日志带来了许多困难。主要有以下几点挑战: 

1.系统越来越多地由多个组件,尤其是分布式组件构成,使用单个日志文件监视系统变得不可能。交叉异构的日志文件很难分析,特别是当时间戳不同步甚至不存在的时候。
2.最大化日志系统的信息量的同时最小化检测成本。
3.单纯的统计模型并不能提供可行性建议。例如,可以用机器学习模型来发现负载中的异常,CPU利用率过高,但是无法解释应该怎么处理。
CCS’2017[5] 提出了一种基于深度学习的日志异常检测系统——DeepLog。DeepLog通过以下三步异常检测来综合判断系统异常.
1.执行路径异常检测。将异常检测问题转换成一个log key的多分类问题,使用LSTM对日志的log key序列建模, 自动从正常的日志数据中学习正常的模式并且由此来判断系统异常。同时,LSTM可以增量式地调整模型参数,以便适应随着时间推移而出现的新日志文件。
2.参数和性能异常检测。有时候系统虽然是按照正常操作步骤执行的,但是记录的日志中的参数是不正常的,比如延迟比正常要大,这种情况也属于异常。
3.工作流异常检测。虽然工作流模型在异常检测的有效性上不如LSTM模型,但是工作流模型可以可视化地帮助运维工程师在发现异常后找出异常的原因。
其中长短期记忆(Long short-term memory, LSTM)是一种特殊的RNN,主要是为了解决长序列训练过程中的梯度消失和梯度爆炸问题。相比普通的RNN,LSTM能够在更长的序列中有更好的表现。

\section{异常检测算法介绍}

本节将介绍在异常检测领域主流的一些算法,根据所依赖的技术原理的不同,将这些算法分为了基于统计、基于分类、基于聚类、基于深度学习的异常检测算法。
\subsection{基于统计的异常检测算法}
早期的异常检测方法往往基于统计与概率模型,也就是假设-检验的方法。首先对数据的分布做出假设,然后找出假设下所定义的“异常”,因此往往使用极值分析或假设检验。
比如对最简单的一维数据假设服从正态分布,然后将距离均值某个范围以外的点当做异常点。推广到高维后,假设各个维度相互独立。这类方法的好处速度一般比较快,但是因为存在很强的“假设”,效果不一定很好。
在时间序列异常检测领域,最常见的基于统计的算法为ARIMA,即差分自回归移动平均模型[8]。

\subsection{基于分类的异常检测算法}
这类方法通常是将异常检测看成是数据不平衡下的分类问题。常用分类算法有朴素贝叶斯、逻辑回归、支持向量机等。
以支持向量机(SVM)为例,SVM在集成学习和神经网络之类的算法没有表现出优越性能前,基本占据了分类模型的统治地位。
目前由于互联网数据规模的急剧膨胀,SVM无法很好的处理海量样本,热度有所下降,但是仍然是一个常用的机器学习算法。
在异常检测领域中,当异常值远远少于正常值时,可以用One-Class SVM。当异常值较多即正负样本均衡时,适用于普通的二分类SVM。可以根据样本情况灵活调整。
One-Class SVM的输入为不包含异常值的“干净”数据,试图求得高维空间的一个超球面,以最小的半径将训练集中的数据包起来。新来的待测数据映射到高维空间后,如果落在这个超球面之外,则认为它是一个异常值。

\subsection{基于聚类的异常检测算法}

聚类算法通常是基于距离/密度发现异常点。基于距离/密度的异常点检测方法的关键步骤在于给每个数据点都分配一个离散度,其主要思想是:
针对给定的数据集,对其中的任意一个数据点,如果在其局部邻域内的点都很密集,那么认为此数据点为正常数据点,而异常点则是距离正常数据点最近邻的点都比较远的数据点。通常有阈值进行界定距离的远近。
异常检测领域下常用的聚类算法有k-means、LOF、孤立森林、高斯混合模型[20]等。

\subsection{基于深度学习的异常检测算法}
随着深度学习的兴起,越来越多的学者尝试用深度学习算法来进行异常检测,尤其是针对时间序列数据,深度学习模型往往表现出惊人的效果。
常用的深度学习算法为变分编码器、神经网络[6][14]、生成对抗网络、LSTM[17]、RNN[3][4][10][12][13][15]等。以变分自动编码器(Variational Auto-Encoder)[5]为例,其利用自编码器的重构误差和局部误差,针对时间序列的异常检测的场景,达到了很好的效果。

\section{异常检测领域开源数据集介绍}
数据集主要由KDDCUP99, CICIDS等。
\section{异常检测算法对比}

\section{现有异常检测算法存在的问题}