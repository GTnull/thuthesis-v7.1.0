% !TeX root = ../thuthesis-example.tex
\chapter{总结与展望}
\section{总结}
随着互联网的规模越来越大,网络流量的规模也急剧膨胀,网络应用的种类日益多样化,因此管理网络的难度也越来越大。传统的基于专家知识、人工生成规则的自动化运维具有很多缺陷:一是无法适应快速变化的网络流量环境,过度依赖规则库,而规则库需要不断人为添加新的规则;二是面对复杂网络环境时,很难根据流量的实时变化动态更新基线。

异常检测作为智能运维的关键环节,现有的异常检测算法具有很多缺陷:面对海量数据规模时,无法或很难做到实时性;应用类型多导致的流量特征复杂,因此很难达到较高的检测率和较低的误报率;大多数异常检测算法仅能报告是否发生异常,难以对确定异常种类和定位异常来源给出指导性意见。

本文以清华大学校园网为例,进行网络流量异常检测算法和系统的研究。清华大学校园网是全球规模最大、架构最复杂、流量场景最多变的校园网之一,具有以下几个特点:
\begin{enumerate}
    \item 用户规模大,流量峰值高。每天有10万台设备活跃在线,同时在线设备最多为7万台,峰值流量约为30Gbps。
    \item	用户应用类型多。清华大学校园网中的用户应用类型远比一般的企业网复杂,在该网络环境中同时存在着几百种应用,这给数据分析带来了很多困难。
    \item	异常流量是常态。扫描流量、攻击流量占比多。
\end{enumerate}

为了应对这些困难和挑战,本文在以下几个方面对流量异常检测进行了研究:
\begin{enumerate}
    \item 本文设计了一种基于特征关系图的循环神经网络算法(FG-RNN)。在复杂的网络流量环境下,流量特征种类繁多,且特征之间的相关性会随着流量变化而变化。原有的异常检测算法通常直接利用提取的特征进行训练,本文提出的FG-RNN算法有效利用了特征之间的相关性信息,将其加入到神经网络的训练过程中。经过在广泛应用的开源数据集、清华大学校园网真实数据集下分别进行的实验验证,本文提出的算法检测结果优于现有的基于机器学习、深度学习的算法。
    \item 本文对基于特征关系图的循环神经网络算法进行了流式改造,使之能够满足当前实时异常检测的需求。为了应对海量的清华大学校园网流量数据和高速数据流,本文设计了一个基于Spark Streaming的实时流量异常检测系统。该系统分为输入模块、模型模块、检测模块三部分,输入模块负责将流量数据进行窗口划分、特征选择、特征抽取、合并计算,得到的特征矩阵与预训练得到的关系矩阵一同送到模型中进行训练,模型中的参数周期性更新,最后由检测模块对当前流量进行判别,并给出异常流量的类别。
    \item 基于该实时异常检测系统,本文在大规模真实数据集上用多个不同的衡量指标对FG-RNN和现有神经网络算法进行实验对比,FG-RNN算法效果均优于其他算法。
    \item 本文对系统检测出的端口扫描进行了定量分析,发现其不仅是安全问题,同时会对无线网性能产生影响,严重时会造成100ms以上的时延增加。最后通过两种优化策略,可以有效地缓解该问题。
\end{enumerate}

\section{未来研究和展望}
本文通过对清华大学校园网流量进行研究,提出了一种基于特征关系图的循环神经网络算法(FG-RNN),并设计和实现了一个基于Spark Streaming的实时异常检测系统。但是其仍然有很多不足之处,本人认为未来的研究可以从以下几个方面着手。
\begin{enumerate}
    \item 本文FG-RNN算法使用的是比较简单的基于皮尔森相关系数的关系矩阵计算方式,该方式并不一定是最佳的关系图计算方式。在未来的工作中,可以尝试其他挖掘方法,如Network Embedding等。
    \item 本文中的异常检测系统只经过了小规模验证,可扩展性需要进一步测试,并且需要更大量的数据和更长时间的运行以验证系统稳定性。
    \item 本文中的算法在处理某些特定异常时,召回率极低,例如无法检测出远控木马这类异常,可以从提升召回率的角度尝试优化模型。
    \item 由于标注样本的缺失,部分异常种类占比很低,因此本文在处理这些特殊异常时效果不佳,未来可以尝试数据增强等方式。
    \item 与商业化的网络运维中心(Network Operation Center)平台中的威胁告警系统相比,基于各种算法的流量检测系统通常人力成本、机器开销更低,但是也有很多不足,如难以涵盖特殊情况,未来可以考虑将两者的优势进行结合。
\end{enumerate}
