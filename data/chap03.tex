% !TeX root = ../thuthesis-example.tex
\chapter{数据集分析与对比}
\section{开源数据集}

\section{校园网真实数据集预处理}

\chapter{基于图结构的RNN及其网络流量异常检测算法}

\section{引言}
Recurrent network的应用主要如下两部分:

文本相关。主要应用于自然语言处理(NLP)、对话系统、情感分析、机器翻译等等领域,Google翻译用的就是一个7-8层的LSTM模型。
时序相关。就是时序预测问题(timeseries),诸如预测天气、温度、包括个人认为根本不可行的但是很多人依旧在做的预测股票价格问题
这些问题都有一个共同点,就是有先后顺序的概念的。举个例子: 根据前5天每个小时的温度,来预测接下来1个小时的温度。典型的时序问题,温度是从5天前,一小时一小时的记录到现在的,它们的顺序不能改变,否则含义就发生了变化;再比如情感分析中,判断一个人写的一篇文章或者说的一句话,它是积极地(positive),还是消极的(negative),这个人说的话写的文章,里面每个字都是有顺序的,不能随意改变,否则含义就不同了。

全连接网络Fully-Connected Network,或者卷积神经网络Convnet,他们在处理一个sequence(比如一个人写的一条影评),或者一个timeseries of data points(比如连续1个月记录的温度)的时候,他们缺乏记忆。一条影评里的每一个字经过word embedding后,被当成了一个独立的个体输入到网络中;网络不清楚之前的,或者之后的文字是什么。这样的网络,我们称为feedforward network。

但是实际情况,我们理解一段文字的信息的时候,每个文字并不是独立的,我们的脑海里也有它的上下文。比如当你看到这段文字的时候,你还记得这篇文章开头表达过一些关于LSTM的信息;

所以,我们在脑海里维护一些信息,这些信息随着我们的阅读不断的更新,帮助我们来理解我们所看到的每一个字,每一句话。这就是RNN的做法:维护一些中间状态信息。
\section{网络流量的时空特性}

\section{基于图结构的RNN原理}

\section{实验方案设计及实验流程}

\section{算法性能评估}

\subsection{基于开源数据集的检测结果}

\subsection{基于真实数据的检测结果}