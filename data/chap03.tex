% !TeX root = ../thuthesis-example.tex

\chapter{数据集分析与对比}
\section{开源数据集}
\subsection{UNSW-NB15}
UNSW-NB15 数据集是 2015 年澳大利亚网络安全中心使用 IXIA Perfect Storm工具模拟网络环境流量而生成的一个数据集。相比于 KDD99 是一个更新的数据集,因此更能代表真实的网络流量。UNSW-NB15 数据集中包括 100GB的.pcap 格式的原始网络流量,同时还有 4 个经过特征提取的 csv 文件,分别是UNSW-NB15\_1.csv  、 UNSW-NB15\_2.csv 、 UNSW-NB15\_3.csv  和   UNSW-NB15\_4.csv,一共是 2540044 条数据,同时该数据集提出了与 KDD99 较为不同的特征,这些特征更为符合当前的网络协议模式。 UNSW-NB15 一共包含有 10 个分类,一个正常类别和 9 个攻击类别,其具体描述和类别数目如下表所示: 
% 表 3.2 攻击类型描述和数目 

\flushleft{

    \begin{table}[H]
        \begin{tabular}{|p{0.1\textwidth}<{\centering} |p{0.7\textwidth}<{\centering} |p{0.1\textwidth}<{\centering}|}
        % \begin{tabular}{lp{3cm}p{9cm}p{3cm}}
        \toprule 
        类别 & 类别描述                                                                       & 样本数量    \\ \midrule 
        Normal  & 正常流量                                                                       & 2218761 \\ \hline
        Fuzzers & 攻击者从命令行或以报文的形式发送大量随机生成的输入序列。攻击者试图发现操作系统、程序或网络中的安全漏洞,并使这些资源挂起一段时间,甚至可以使它们崩溃 & 24246   \\ \hline
        Analysis & 这类攻击是指通过端口扫描、恶意 web 脚本
    (如 HTML 文件渗透)和发送垃圾邮件等各种
    方式渗透到 web 应用程序的各种入侵等。 & 2677 \\ \hline
    Backdoor & 这类攻击中攻击者可以绕过正常的身份验证
    并获得对系统的未授权远程访问。黑客利用
    后门程序安装恶意文件,修改代码或获得对
    系统或数据的访问。 &
    2329 \\ \hline
    DoS & 攻击者使某些计算或内存资源过于繁忙或占
    据全部资源而无法处理合法请求或者拒绝合
    法用户对计算机的访问。 &
    16353 \\ \hline
    Exploit & 利用操作系统或软件中的软件漏洞、漏洞或
    故障进行入侵的行为。攻击者利用软件的知
    识发动攻击,意图对系统造成危害。 &
    44525 \\ \hline
    Generic & 针对密码系统的攻击,试图破坏安全系统的
    密钥。它独立于密码系统的实现细节。不考
    虑块密码的结构。例如,生日攻击是一种将
    哈希函数视为黑盒的通用攻击。 &
    215481 \\ \hline
    Reconnaissance & 为了绕过目标计算机网络的安全控制而收集
    其信息的攻击。它可以被定义为一个探针,
    是发起进一步攻击的初步步骤。攻击者使用
    各种扫描手段来收集系统信息。在收集到足
    够的信息后,可以发起后续的攻击。 &
    13987 \\ \hline
    Shellcode & Shellcode 作为负载在目标机器执行,来挖掘
    该软件的漏洞。之所以称作 Shellcode 是因为
    启动了受到攻击者控制的命令行 shell。 &
    1511 \\ \hline
    Worm & 蠕虫是一种恶意程序或恶意软件,它可以复
    制自己并传播到其他计算机。 &
    174 \\  
        \bottomrule
        \end{tabular}
        \end{table}

}


 
UNSW-NB15 数据集一共包含 47 个特征,其中时间戳,IP 地址,端口号等特
征对训练无用,因此有效的特征一共 41 个。 
下面对这些特征做一个概括的说明。按照数据集作者的思路,可以分为基本
特征,内容特征,时间特征和额外生成的特征这几类。这里从另外一种思路进行
重新归类可以分为以下几类。


\begin{table}[H]
    \caption{与协议相关的特征}
    \centering
    \begin{tabular}{|l|l|}
    \hline
    特征名称                  & 特征描述                      \\ \hline
    proto                 & 传输层协议                     \\ \hline
    service               & 应用层协议                     \\ \hline
    sttl                  & 从源发出的报文的 time to live     \\ \hline
    dttl                  & 从目的发出的报文的 time to live 字段 \\ \hline
    stcpb                 & 源 tcp 报文的初始序列号            \\ \hline
    dtcpb                 & 目的 tcp 报文的初始序列号           \\ \hline
    swin                  & 源 tcp 报文的窗口字段             \\ \hline
    dwin                  & 目的 tcp 报文的窗口字段            \\ \hline
    res\_bdy\_len         & http 响应内容的长度              \\ \hline
    ct\_flw\_http\_method & 会话中 http 方法字段的计数          \\ \hline
    is\_ftp\_login        & 是否有 ftp 的登录               \\ \hline
    ct\_ftp\_cmd          & 会话中 ftp 命令的计数             \\ \hline
    trans\_depth          & http 服务的连接深度              \\ \hline
    \end{tabular}
    \end{table}

\begin{table}[H]
    \caption{与时间相关的特征}
    \centering
    \begin{tabular}{|l|l|}
    \hline
    dur     & 会话的持续时间                        \\ \hline
    tcprtt  & tcp 三次握手的 rtt(round trip time) \\ \hline
    synack  & 第一次发送到确认的时间                    \\ \hline
    ackdat  & 确认之后返回的时间                      \\ \hline
    sintpkt & 源报文的间隔时间的平均值                   \\ \hline
    dintpkt & 目的报文间隔时间的平均值                   \\ \hline
    sjit    & 源报文间隔时间的标准差(jitter)            \\ \hline
    djit    & 目的报文间隔时间的标准差                   \\ \hline
    sload   & 源报文的吞吐量                        \\ \hline
    dload   & 目的报文的吞吐量                       \\ \hline
    \end{tabular}
    \end{table}

    \begin{table}[H]
        \caption{与报文大小相关的特征}
        \centering
        \begin{tabular}{|l|l|}
        \hline
        sbytes  & 会话中从源发出的总字节数  \\ \hline
        dbytes  & 会话中从目的发出的总字节数 \\ \hline
        smeansz & 会话中源报文的平均大小   \\ \hline
        dmeansz & 会话中目的报文的平均大小  \\ \hline
        spkts   & 会话中源的报文总数     \\ \hline
        dpkts   & 会话中目的报文总数     \\ \hline
        \end{tabular}
        \end{table}

\begin{table}[H]
    \caption{与连接状态相关的特征}
    \centering
    \begin{tabular}{|l|l|}
    \hline
    sloss & 源的丢包数和重传数之和  \\ \hline
    dloss & 目的的丢包数和重传数之和 \\ \hline
    state & 会话的状态和相应的协议  \\ \hline
    \end{tabular}
    \end{table}

\begin{table}[H]
    \caption{额外构造的特征}
    \centering
    \begin{tabular}{|l|l|}
    \hline
    ct\_srv\_src                & 根据最后一条报文的时间排序,每 100 条记录中源ip 与服务都相同的会话计数(下面省略每 100 条)  \\ \hline
    ct\_srv\_dst                & 目的 IP 与服务都相同的会话计数         \\ \hline
    ct\_dst\_ltm                & 目的 IP 相同的会话计数             \\ \hline
    ct\_src\_ltm                & 源 IP 相同的会话计数              \\ \hline
    ct\_src\_dport\_ltm         & 源 IP 和目的端口都相同的会话计数        \\ \hline
    ct\_dst\_sport\_ltm         & 目的 IP 和源端口都相同的会话计数        \\ \hline
    ct\_dst\_src\_ltm           & 目的 IP 和源 IP 都相同的会话计数      \\ \hline
    ct\_state\_ttl              & 对于每一个状态,ttl 值的范围          \\ \hline
    is\_sm\_ips\_ports          & 源 ip 与目的 ip,源端口和目的端口是否都相同 \\ \hline
    \end{tabular}
    \end{table}
 
可以看到,UNSW-NB15 数据集大多数特征都有较为清晰的定义,因此可以
用做流量数据特征提取的标准。 

\subsection{CICIDS2017}
CICIDS数据集特征介绍如下:(修改成跨页表格)


List of extracted features and descriptions: \\

% \begin{longtable}{cc}
%     \begin{tabular}
%       abc
% \end{tabular}
% \end{longtable}

Flow duration			Duration of the flow in Microsecond \\
total Fwd Packet		Total packets in the forward direction \\
total Bwd packets		Total packets in the backward direction \\
total Length of Fwd Packet	Total size of packet in forward direction \\
total Length of Bwd Packet	Total size of packet in backward direction \\
Fwd Packet Length Min 		Minimum size of packet in forward direction \\
Fwd Packet Length Max 		Maximum size of packet in forward direction \\
Fwd Packet Length Mean		Mean size of packet in forward direction \\
Fwd Packet Length Std		Standard deviation size of packet in forward direction \\
Bwd Packet Length Min		Minimum size of packet in backward direction \\
Bwd Packet Length Max		Maximum size of packet in backward direction \\
Bwd Packet Length Mean		Mean size of packet in backward direction \\
Bwd Packet Length Std		Standard deviation size of packet in backward direction \\
Flow Bytes/s			Number of flow bytes per second \\
Flow Packets/s			Number of flow packets per second  \\
Flow IAT Mean			Mean time between two packets sent in the flow \\
Flow IAT Std			Standard deviation time between two packets sent in the flow \\
Flow IAT Max			Maximum time between two packets sent in the flow \\
Flow IAT Min			Minimum time between two packets sent in the flow \\
Fwd IAT Min			Minimum time between two packets sent in the forward direction \\
Fwd IAT Max			Maximum time between two packets sent in the forward direction \\
Fwd IAT Mean			Mean time between two packets sent in the forward direction \\
Fwd IAT Std			Standard deviation time between two packets sent in the forward direction \\
Fwd IAT Total   		Total time between two packets sent in the forward direction \\
Bwd IAT Min			Minimum time between two packets sent in the backward direction \\
Bwd IAT Max			Maximum time between two packets sent in the backward direction \\
Bwd IAT Mean			Mean time between two packets sent in the backward direction \\
Bwd IAT Std			Standard deviation time between two packets sent in the backward direction \\
Bwd IAT Total			Total time between two packets sent in the backward direction \\
Fwd PSH flags			Number of times the PSH flag was set in packets travelling in the forward direction (0 for UDP) \\
Bwd PSH Flags			Number of times the PSH flag was set in packets travelling in the backward direction (0 for UDP) \\
Fwd URG Flags			Number of times the URG flag was set in packets travelling in the forward direction (0 for UDP) \\
Bwd URG Flags			Number of times the URG flag was set in packets travelling in the backward direction (0 for UDP) \\
Fwd Header Length		Total bytes used for headers in the forward direction \\
Bwd Header Length		Total bytes used for headers in the backward direction \\
FWD Packets/s			Number of forward packets per second \\
Bwd Packets/s			Number of backward packets per second \\
Packet Length Min 		Minimum length of a packet \\
Packet Length Max		Maximum length of a packet \\
Packet Length Mean 		Mean length of a packet \\
Packet Length Std		Standard deviation length of a packet \\
Packet Length Variance  	Variance length of a packet \\
FIN Flag Count 			Number of packets with FIN \\
SYN Flag Count 			Number of packets with SYN \\
RST Flag Count 			Number of packets with RST \\
PSH Flag Count 			Number of packets with PUSH \\
ACK Flag Count 			Number of packets with ACK \\
URG Flag Count 			Number of packets with URG \\
CWR Flag Count 			Number of packets with CWR \\
ECE Flag Count 			Number of packets with ECE \\
down/Up Ratio			Download and upload ratio \\
Average Packet Size 		Average size of packet \\
Fwd Segment Size Avg 		Average size observed in the forward direction \\
Bwd Segment Size Avg 		Average number of bytes bulk rate in the backward direction \\
Fwd Bytes/Bulk Avg		Average number of bytes bulk rate in the forward direction \\
Fwd Packet/Bulk Avg		Average number of packets bulk rate in the forward direction \\
Fwd Bulk Rate Avg 		Average number of bulk rate in the forward direction \\
Bwd Bytes/Bulk Avg		Average number of bytes bulk rate in the backward direction \\
Bwd Packet/Bulk Avg 		Average number of packets bulk rate in the backward direction \\
Bwd Bulk Rate Avg		Average number of bulk rate in the backward direction \\
Subflow Fwd Packets		The average number of packets in a sub flow in the forward direction \\
Subflow Fwd Bytes		The average number of bytes in a sub flow in the forward direction \\
Subflow Bwd Packets		The average number of packets in a sub flow in the backward direction \\
Subflow Bwd Bytes		The average number of bytes in a sub flow in the backward direction \\
Fwd Init Win bytes		The total number of bytes sent in initial window in the forward direction \\
Bwd Init Win bytes		The total number of bytes sent in initial window in the backward direction \\
Fwd Act Data Pkts		Count of packets with at least 1 byte of TCP data payload in the forward direction \\
Fwd Seg Size Min		Minimum segment size observed in the forward direction \\
Active Min			Minimum time a flow was active before becoming idle \\
Active Mean			Mean time a flow was active before becoming idle \\
Active Max			Maximum time a flow was active before becoming idle \\
Active Std			Standard deviation time a flow was active before becoming idle \\
Idle Min			Minimum time a flow was idle before becoming active \\
Idle Mean			Mean time a flow was idle before becoming active \\
Idle Max			Maximum time a flow was idle before becoming active \\
Idle Std			Standard deviation time a flow was idle before becoming active \\
\section{校园网真实数据集}
