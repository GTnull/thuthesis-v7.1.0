% !TeX root = ../thuthesis-example.tex

\chapter{指导教师学术评语}

面向智能运维的流量异常检测是目前网络研究领域的开放性问题。论文选题基于校园网真实运行环境,实现校园网流量异常检测,具有一定的理论意义和现实价值。

论文完成的主要工作包括:
\begin{enumerate}
    \item 提出了一种基于特征关系图的循环神经网络算法,通过引入特征关系图,进一步强化多维数据中的重要维度数据对机器学习效果的影响,提高流量异常检测算法的准确率和召回率。
    \item 基于该神经网络算法,针对校园网真实流量环境,设计和实现了基于Spark-Streaming的异常流量检测系统,该系统通过引入特征关系图更新机制来避免传统机器学习算法的时效性问题。
    \item 论文还对流量异常检测过程中发现的扫描攻击行为进行了分析,发现了扫描攻击行为对无线网络性能的影响,并提出了针对扫描攻击的防御策略。
\end{enumerate}
% 1、提出了一种基于特征关系图的循环神经网络算法,通过引入特征关系图,进一步强化多维数据中的重要维度数据对机器学习效果的影响,提高流量异常检测算法的准确率和召回率。
% 2、基于该神经网络算法,针对校园网真实流量环境,设计和实现了基于Spark-Streaming的异常流量检测系统,该系统通过引入特征关系图更新机制来避免传统机器学习算法的时效性问题。
% 3、论文还对流量异常检测过程中发现的扫描攻击行为进行了分析,发现了扫描攻击行为对无线网络性能的影响,并提出了针对扫描攻击的防御策略。
论文工作表明,作者在本学科掌握了较为坚实的基础理论和系统的专门知识,具有独立承担专门技术工作的能力。论文论述清晰,表述规范,达到了硕士学位论文水平。

建议组织硕士学位论文答辩。

